Welcome to the Retro Graphics Toolkit offline manual. This manual contain
information on using Retro Graphics Toolkit. It is recommended that you
read these pages to get a better grasp of Retro Graphics Toolkit.
Seeing the page count may cause a shock however easy page is short and easy to read.
If anything needs clarified you can always contact me by opening
up an issue, fixing the wiki yourself (which will in turn make it in this offline version) or reply to the Retro Graphics
Toolkit forum topic.

There are a few programs that will import art for various Retro consoles
but when it comes to converting art to currently supported systems no
program does it better than Retro Graphics Toolkit.

Retro Graphics Toolkit was designed from the ground up to make managing
and importing art to various gaming consoles and embedded devices a
breeze. With its easy to use GUI you can jump right in and if you need
help I can answer questions or you can read the tutorials that are here
on the wiki

\subsection{Features}\label{features}

Retro graphics toolkit has a rich feature set some of which are listed
here and includes

\begin{itemize}
\itemsep1pt\parskip0pt\parsep0pt
\item
  Importing common file formats such as png,jpg,bmp,tiff and more\\
\item
  Several dithering choices and nearest color\\
\item
  Multi-platform both in the sense that this runs on multiple operating
  systems and in the sense that you can create art for multiple
  systems.\\
\item
  True color workflow allowing for easy changes in palette (to find out
  more read the article)\\
\item
  Open source (GPLv3 licensed)\\
\item
  Can easily make changes to tiles just by selecting a color and
  clicking on it.\\
\item
  Two easy ways to place and modify tile placement. Either left click
  and the tile will placed on the current location using attributes that
  you selected or right click on the tile and then when you use the
  buttons it will affect tile with blue rectangle around it.\\
\item
  Supports common compression featured in many sega genesis games.\\
\item
  Good handling of alpha transparency. When you import an image with
  alpha transparency this is preserved and dithered.\\
\item
  Tilemap blocks,chunks and advanced sprite editor
\end{itemize}

If that sounds like something you are interested in either compile the
source code or if you are a windows user I supply up to date windows
binaries here:
\url{https://github.com/ComputerNerd/Retro-Graphics-Toolkit/blob/master/RetroGraphicsToolkit.exe.7z}
